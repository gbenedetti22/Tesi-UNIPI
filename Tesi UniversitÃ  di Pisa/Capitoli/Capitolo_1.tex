\chapter{Introduzione}
Il presente elaborato descrive il tirocinio svolto presso l'azienda 01 Informatica\cite{01info} a Pescia durante i mesi di Luglio e Agosto. L'azienda è nota per il suo impegno nel settore dello sviluppo e del mantenimento software di alta qualità, destinato a una clientela prettamente aziendale. Durante il periodo di tirocinio, mi è stato offerto di partecipare al progetto di sviluppo, design e realizzazione di un'applicazione mobile (denominata Companion) simile a una piattaforma di e-commerce, destinata a sistemi operativi Android e iOS.\\
L'obiettivo principale del progetto è stato quello di fornire agli utenti un'esperienza di "shopping virtuale" agevole e personalizzata, consentendo loro, previa esecuzione del login o accesso tramite deeplink, di acquistare una vasta gamma di prodotti, i quali possono essere aggiunti ad un carrello e, successivamente, inoltrati per l'acquisto.\\
Durante lo sviluppo di questa applicazione, è stato necessario focalizzarsi su due fondamentali componenti: il Server e il Client. Entrambi questi aspetti sono stati oggetto di un'attenta progettazione e implementazione, con l'intero processo sviluppativo affidato al sottoscritto come tirocinante presso l'azienda in questione. La sfida principale è stata quella di creare un sistema perfettamente integrato, in grado di gestire una vasta quantità di dati e transazioni, garantendo al contempo sicurezza, affidabilità e prestazioni ottimali.\\
Nel corso dell'elaborato verranno presentati i dettagli dell'architettura dell'applicazione, i metodi e le tecnologie utilizzate (sia lato Server che lato Client), si descriveranno le sfide incontrate e le scelte progettuali effettuate, illustrando i risultati ottenuti e analizzando i possibili miglioramenti futuri.

\section{Motivazioni sullo sviluppo dell'app}
La scelta di concentrarsi sull'implementazione di un'applicazione mobile è stata motivata dalla crescente diffusione e importanza degli smartphone nel panorama tecnologico contemporaneo. Tali dispositivi, infatti, hanno radicalmente trasformato le abitudini di consumo, aprendo nuove prospettive nel settore dell'e-commerce e spingendo le aziende a investire sempre di più nello sviluppo di soluzioni innovative per il mercato mobile.\\
Nello specifico, lo sviluppo di questa applicazione ha trovato le sue radici nell'esigenza di risolvere una sfida importante per le aziende, desiderose di fornire ai propri clienti un'esperienza di acquisto agevole e altamente efficiente all'interno di un catalogo prodotti pre-selezionato. Pertanto, l'applicazione è stata concepita come un "compagno" (da cui deriva appunto il suo nome "Companion") in grado di affiancare, assistere e integrare sia l'ampia gamma di prodotti già presenti sia gli utenti registrati nei Database dell'azienda. Un aspetto peculiare e innovativo, che sarà oggetto di ulteriore approfondimento, riguarda l'implementazione dei deeplink: mediante l'uso di questi collegamenti diretti, l'azienda è in grado di inviare al proprio cliente un link personalizzato, attraverso il quale è possibile automatizzare determinate operazioni (ad esempio, l'aggiunta di un prodotto specifico al carrello), rendendo l'applicazione accessibile anche ai meno esperti.\\
La riuscita nell'utilizzo di questi deeplink ha notevolmente agevolato il processo di ricerca e acquisto, permettendo ai clienti di essere guidati in modo rapido e automatico a selezionare i prodotti di interesse, senza richiedere una conoscenza approfondita dell'applicazione stessa.

\section{Guida alla relazione}
Nel presente capitolo è stata descritta una panoramica generale del contesto di svolgimento del tirocinio e del progetto proposto dall'azienda, con particolare enfasi sulle motivazioni che hanno spinto lo sviluppo dell'applicazione. Il Capitolo 2 introdurrà le tecnologie impiegate, mentre il Capitolo 3 fornirà una descrizione approfondita del processo di sviluppo dell'applicazione. Infine nel Capitolo 4 saranno presentate le considerazioni finali riguardanti l'esperienza di tirocinio, nonché le possibili prospettive di sviluppo futuro.
