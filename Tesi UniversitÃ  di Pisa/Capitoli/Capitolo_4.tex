\chapter{Conclusioni}
In questo documento si è delineato l'intero processo di sviluppo di un'applicazione mobile, sia lato back-end che front-end, che fa uso sia di approcci più tradizionali sia sfruttando l'utilizzo dei deeplink. L'obiettivo principale è stato, quindi, la creazione di un'applicazione, denominata Companion, che consentisse agli utenti di effettuare acquisti in modo agevole, sfruttando questo approccio ibrido: da un lato, sono state implementate funzionalità più tradizionali, come la ricerca e l'acquisto mediante l'uso di una barra di ricerca e di un carrello, mentre dall'altro si è sfruttato l'uso dei deeplink come dettagliato nella \Cref{sub:deeplink}, al fine di facilitare l'esperienza utente, in quanto consentono agli utenti di acquistare i prodotti dal catalogo cliccando su un semplice URL. Si sottolinea che ogni utente ha avuto accesso a un listino prezzi dedicato, come descritto esaustivamente nella \Cref{section2}.
Per quanto riguarda il lato back-end, il Server è stato implementato con una gestione multithreading in grado di gestire contemporaneamente multiple richieste. Questo è stato realizzato utilizzando la libreria Javalin, come descritto nella \Cref{subsub:javalin} e l'adozione dei JSON Web Token, che hanno reso il sistema altamente scalabile in quanto le informazioni sugli utenti non sono memorizzate direttamente sul disco, ma dentro questi token come ampiamente descritto nella sezione \Cref{section_token}.\\
Companion è quindi un'applicazione su misura, sviluppata per soddisfare le specifiche esigenze di un cliente in particolare.
Un obiettivo significativo raggiunto riguarda l'adozione di Javalin come framework lato Server: l'ottimizzazione delle risorse di memoria ha permesso di ottenere un'applicazione reattiva, leggera e, inoltre, la persistenza su disco si è dimostrata preziosa per garantire la continuità dell'applicazione anche in caso di chiusura inattesa del Server (vedi \Cref{subsub:key_creation_and_disk_persistency}).
Nel complesso, si è cercato di adottare un approccio efficiente nello sviluppo, scegliendo algoritmi con complessità ottimale e evitando scenari di tempo computazionalmente elevato ($O(n^2)$). Inoltre, lato Client, è stata ottimizzata l'efficienza minimizzando la ricostruzione dei Widget quando non strettamente necessario.

\section{Funzionalità future e possibili miglioramenti}
Essendo Companion un'applicazione orientata alla facilità di acquisto dei prodotti, alcune funzionalità aggiuntive possono essere: una cronologia degli acquisti, in grado di mostrare agli utenti cosa hanno comprato in passato (e quindi di poterle ricomprare), la capacità di scannerizzare con la fotocamera un codice a barre personalizzato creato dall'azienda, al fine di aggiungere automaticamente un prodotto al carrello e, infine, la possibilità di inserire manualmente la quantità di un prodotto senza l'ausilio dei pulsanti nella pagina del carrello.\\
Come accennato in precedenza, la classe 'Provider' è stata utilizzata in modo estensivo per facilitare la comunicazione tra i Widget; tuttavia, con l'esperienza accumulata, ritengo che in futuro sia necessario optare per alternative diverse a questo framework, poiché è stato riscontrato che il suo utilizzo può risultare tedioso in diverse circostanze, sopratutto riguardo alla necessità di avere un "context", ovvero un oggetto chiave di Flutter per accedere all'oggetto globale e questa dipendenza può portare a molteplici verifiche quando si gestiscono chiamate asincrone, in quanto questo oggetto potrebbe desistere in qualunque momento.
Lato Server, l'esperienza con Javalin è stata stimolante e di semplice comprensione, tuttavia, in scenari reali, probabilmente sarebbe preferibile utilizzare framework più robusti ed efficienti, dato che esistono alcune funzionalità essenziali mancanti di completezza (per esempio, il supporto a HTTPS non è pienamente supportato)

\section{Considerazioni sul tirocinio}
Nello svolgimento del tirocinio, sono stati appresi diversi strumenti che durante il percorso universitario non è stato possibile conoscere. In particolare, l'uso dei token JWT e il linguaggio Flutter sono le due principali caratteristiche che sono state apprezzate di più, in quanto Flutter permette di scrivere un applicazione una volta e averla su più piattaforme, mentre JWT ha il grande vantaggio di facilitare la reperibilità dei dati importanti.\\ Il tirocinio in generale ha permesso quindi di conoscere tecnologie nuove e di poter sviluppare un applicativo effettivamente usabile da un azienda per poter vendere ai suoi clienti i propri prodotti.